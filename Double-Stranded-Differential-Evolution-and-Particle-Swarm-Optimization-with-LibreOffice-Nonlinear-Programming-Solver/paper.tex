\documentclass[runningheads]{llncs}

\usepackage[T1]{fontenc}

\usepackage{graphicx}

%\usepackage{color}
%\renewcommand\UrlFont{\color{blue}\rmfamily}

\begin{document}

\title{Double-Stranded Differential Evolution and Particle Swarm Optimization with LibreOffice Nonlinear Programming Solver\thanks{Supported by organization Velbazhd Software LLC.}}

\titlerunning{Double-Stranded DE\&PSO with LO NPS}

\author{
Gergana Mateeva \and
Ivan Blagoev \and\\
Kalin Kopanov \and
Velizar Varbanov \and\\
Todor Balabanov\orcidID{0000-0003-3139-069X}
}

\authorrunning{G. Mateeva et al.}

\institute{
Bulgarian Academy of Sciences\\ Institute of Information and Communication Technologies\\ acad. Georgi Bonchev Str., block 2, 1113 Sofia, Bulgaria
\email{\{gergana.mateeva,ivan.blagoev,\\kalin.kopanov,velizar.varbanov,\\todor.balabanov\}@iict.bas.bg}\\
\url{http://www.iict.bas.bg/}
}

\maketitle

\begin{abstract}
Differential Evolution and Particle Swarm Optimization are heuristic global optimization methods inspired by natural evolution and swarm behavior. They are often used to solve complex optimization and simulation problems that are time-consuming or impossible to solve using exact numerical methods. Traditionally, RNA ideas are closer to Differential Evolution population formation. This paper proposes a double-stranded (more DNA-like) implementation of population in LibreOffice Calc NLP Solver. The proposed implementation is validated with well-known optimization benchmark functions.

\keywords{Double-stranded genetic algorithms \and Nonlinear optimization.}
\end{abstract}

\section{Introduction}

\subsection{Selection in Population-based Algorithms}

\subsection{Crossover in Population-based Algorithms}

\subsection{Mutation in Population-based Algorithms}

\subsection{Natural Genetics}

\section{Double-Stranded Candidate Solutions}

\section{Pragmatic Realization}

\section{Experiments and Results}

\section{Conclusions}

\section*{Acknowledgements} This research is partially supported by the ...

\begin{thebibliography}{8}
\bibitem{}
\end{thebibliography}
\end{document}
